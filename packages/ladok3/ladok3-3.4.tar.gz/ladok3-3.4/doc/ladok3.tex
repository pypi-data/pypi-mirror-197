\documentclass[a4paper,oneside]{book}
\newenvironment{abstract}{}{}
\usepackage{abstract}
\usepackage{noweb}
% Needed to relax penalty for breaking code chunks across pages, otherwise 
% there might be a lot of space following a code chunk.
\def\nwendcode{\endtrivlist \endgroup}
\let\nwdocspar=\smallbreak

\usepackage[hyphens]{url}
\usepackage{hyperref}
\usepackage{authblk}

% package with esp-idf specific macros
\usepackage{espidf}

\setcounter{secnumdepth}{2}
\setcounter{tocdepth}{2}

\usepackage{amsmath,amsfonts,amssymb,amsthm}
\usepackage{graphicx}
%%% reduce spaces for Table of contents, figures and tables
%%% it is used "\addtocontents{toc}{\vskip -1.2cm}" etc. in the document
\usepackage[notlot,nottoc,notlof]{}

\usepackage{color}
\usepackage{transparent}
\usepackage{eso-pic}
\usepackage{lipsum}

%%% Needed for displaying Chinese in English documentation
\usepackage{xeCJK}

%% spacing between line
\usepackage{setspace}
\singlespacing


\definecolor{myred}{RGB}{229, 32, 26}
\definecolor{mygrayy}{RGB}{127, 127, 127}
\definecolor{myblack}{RGB}{64, 64, 64}


%%%%%%%%%%% datetime
\usepackage{datetime}

\newdateformat{MonthYearFormat}{%
    \monthname[\THEMONTH], \THEYEAR}


%% RO, LE will not work for 'oneside' layout.
%% Change oneside to twoside in document class
\usepackage{fancyhdr}
\pagestyle{fancy}
\fancyhf{}

% Header and footer
\makeatletter
  \fancypagestyle{normal}{
    \fancyhf{}
    \fancyhead[L]{\nouppercase{\leftmark}}
    \fancyfoot[C]{\py@HeaderFamily\thepage \\ \href{https://www.espressif.com/en/company/documents/documentation_feedback?docId=4287&sections=&version=\idfReleaseName}{Submit Document Feedback}}
    \fancyfoot[L]{Espressif Systems}
    \fancyfoot[R]{\idfReleaseName}
    \renewcommand{\headrulewidth}{0.4pt}
    \renewcommand{\footrulewidth}{0.4pt}
  }
\makeatother

\renewcommand{\headrulewidth}{0.5pt}
\renewcommand{\footrulewidth}{0.5pt}


% Define a spacing for section, subsection and subsubsection
% http://tex.stackexchange.com/questions/108684/spacing-before-and-after-section-titles

\titlespacing*{\section}{0pt}{6pt plus 0pt minus 0pt}{6pt plus 0pt minus 0pt}
\titlespacing*{\subsection}{0pt}{18pt plus 64pt minus 0pt}{0pt}
\titlespacing*{\subsubsection}{0pt}{12pt plus 0pt minus 0pt}{0pt}
\titlespacing*{\paragraph}    {0pt}{3.25ex plus 1ex minus .2ex}{1.5ex plus .2ex}
\titlespacing*{\subparagraph} {0pt}{3.25ex plus 1ex minus .2ex}{1.5ex plus .2ex}

% Define the colors of table of contents
% This is helpful to understand http://tex.stackexchange.com/questions/110253/what-the-first-argument-for-lsubsection-actually-is
\definecolor{LochmaraColor}{HTML}{1020A0}

% Hyperlinks
\hypersetup{
    colorlinks = true,
    allcolors = {LochmaraColor},
}


\RequirePackage{tocbibind} %%% comment this to remove page number for following
\addto\captionsenglish{\renewcommand{\contentsname}{Table of contents}}
\addto\captionsenglish{\renewcommand{\listfigurename}{List of figures}}
\addto\captionsenglish{\renewcommand{\listtablename}{List of tables}}
% \addto\captionsenglish{\renewcommand{\chaptername}{Chapter}}




%%reduce spacing for itemize
\usepackage{enumitem}
\setlist{nosep}

%%%%%%%%%%% Quote Styles at the top of chapter
\usepackage{epigraph}
\setlength{\epigraphwidth}{0.8\columnwidth}
\newcommand{\chapterquote}[2]{\epigraphhead[60]{\epigraph{\textit{#1}}{\textbf {\textit{--#2}}}}}
%%%%%%%%%%% Quote for all places except Chapter
\newcommand{\sectionquote}[2]{{\quote{\textit{``#1''}}{\textbf {\textit{--#2}}}}}

% Insert 22pt white space before roc title. \titlespacing at line 65 changes it by -22 later on.
\renewcommand*\contentsname{\hspace{0pt}Contents}


% Define section, subsection and subsubsection font size and color
\usepackage{sectsty}
\definecolor{AllportsColor}{HTML}{A02010}
\allsectionsfont{\color{AllportsColor}}

\usepackage{titlesec}
\titleformat{\section}
{\color{AllportsColor}\LARGE\bfseries}{\thesection.}{1em}{}

\titleformat{\subsection}
{\color{AllportsColor}\Large\bfseries}{\thesubsection.}{1em}{}

\titleformat{\subsubsection}
{\color{AllportsColor}\large\bfseries}{\thesubsubsection.}{1em}{}

\titleformat{\paragraph}
{\color{AllportsColor}\large\bfseries}{\theparagraph}{1em}{}

\titleformat{\subparagraph}
  {\normalfont\normalsize\bfseries}{\thesubparagraph}{1em}{}

\titleformat{\subsubparagraph}
  {\normalfont\normalsize\bfseries}{\thesubsubparagraph}{1em}{}


\title{%
  A LADOK3 Python API
}
\author{%
  Alexander Baltatzis,
  Daniel Bosk,
  Gerald Q.\ \enquote{Chip} Maguire Jr.
}
\affil{%
  KTH EECS\\
  \texttt{\{alba,dbosk,maguire\}@kth.se}
}

\begin{document}
\frontmatter
\maketitle

\vspace*{\fill}
\VerbatimInput{../LICENSE}
\clearpage

\begin{abstract}
  We provide a Python wrapper for the LADOK3 REST API\@.
We provide a useful object-oriented framework and direct API calls that return 
the unprocessed JSON data from LADOK.

\end{abstract}
\clearpage

\tableofcontents
\clearpage

\mainmatter
\chapter{Introduction}

LADOK (abbreviation of \foreignlanguage{swedish}{Lokalt Adb–baserat 
DOKumentationssystem}, in Swedish) is the national documentation system for 
higher education in Sweden.
This is the documented source code of \texttt{ladok3}, a LADOK3 API wrapper for 
Python.

The \texttt{ladok3} library provides a non-GUI application that, similar to 
access via a web browser, only provides the user with access to the LADOK data 
and functionality that they would actually have based on their specific user 
permissions in LADOK.
It can be seem as a very streamlined web browser just for LADOK's web 
interface.
While the library exploits caching to reduce the load on the LADOK server, this 
represents a subset of the information that would otherwise be obtained via
LADOK's web GUI export functions.

You can install the \texttt{ladok3} package by running
\begin{minted}{bash}
pip3 install ladok3
\end{minted}
in the terminal.
You can find a quick reference by running
\begin{minted}[firstnumber=last]{bash}
pydoc ladok3
\end{minted}

We provide the main class \texttt{LadokSession} (\cref{LadokSession}).
The \texttt{LadokSession} class acts like \enquote{factories} and will return 
objects representing various LADOK data.
These data objects' classes inherit the \texttt{LadokData} (\cref{LadokData}) 
and \texttt{LadokRemoteData} (\cref{LadokRemoteData}) classes.
Data of the type \texttt{LadokData} is not expected to change, unlike 
\texttt{LadokRemoteData}, which is.
Objects of type \texttt{LadokRemoteData} know how to update themselves, \ie fetch 
and refresh their data from LADOK.
When relevant they can also write data to LADOK, \ie update entries such as 
results.

One design criteria is to improve performance.
We do this by caching all factory methods of any \texttt{LadokSession}.
The \texttt{LadokSession} itself is also designed to be cacheable: if the session to 
LADOK expires, it will automatically reauthenticate to establish a new session.



\part{The library}

\documentclass[a4paper,oneside]{book}
\newenvironment{abstract}{}{}
\usepackage{abstract}
\usepackage{noweb}
% Needed to relax penalty for breaking code chunks across pages, otherwise 
% there might be a lot of space following a code chunk.
\def\nwendcode{\endtrivlist \endgroup}
\let\nwdocspar=\smallbreak

\usepackage[hyphens]{url}
\usepackage{hyperref}
\usepackage{authblk}

% package with esp-idf specific macros
\usepackage{espidf}

\setcounter{secnumdepth}{2}
\setcounter{tocdepth}{2}

\usepackage{amsmath,amsfonts,amssymb,amsthm}
\usepackage{graphicx}
%%% reduce spaces for Table of contents, figures and tables
%%% it is used "\addtocontents{toc}{\vskip -1.2cm}" etc. in the document
\usepackage[notlot,nottoc,notlof]{}

\usepackage{color}
\usepackage{transparent}
\usepackage{eso-pic}
\usepackage{lipsum}

%%% Needed for displaying Chinese in English documentation
\usepackage{xeCJK}

%% spacing between line
\usepackage{setspace}
\singlespacing


\definecolor{myred}{RGB}{229, 32, 26}
\definecolor{mygrayy}{RGB}{127, 127, 127}
\definecolor{myblack}{RGB}{64, 64, 64}


%%%%%%%%%%% datetime
\usepackage{datetime}

\newdateformat{MonthYearFormat}{%
    \monthname[\THEMONTH], \THEYEAR}


%% RO, LE will not work for 'oneside' layout.
%% Change oneside to twoside in document class
\usepackage{fancyhdr}
\pagestyle{fancy}
\fancyhf{}

% Header and footer
\makeatletter
  \fancypagestyle{normal}{
    \fancyhf{}
    \fancyhead[L]{\nouppercase{\leftmark}}
    \fancyfoot[C]{\py@HeaderFamily\thepage \\ \href{https://www.espressif.com/en/company/documents/documentation_feedback?docId=4287&sections=&version=\idfReleaseName}{Submit Document Feedback}}
    \fancyfoot[L]{Espressif Systems}
    \fancyfoot[R]{\idfReleaseName}
    \renewcommand{\headrulewidth}{0.4pt}
    \renewcommand{\footrulewidth}{0.4pt}
  }
\makeatother

\renewcommand{\headrulewidth}{0.5pt}
\renewcommand{\footrulewidth}{0.5pt}


% Define a spacing for section, subsection and subsubsection
% http://tex.stackexchange.com/questions/108684/spacing-before-and-after-section-titles

\titlespacing*{\section}{0pt}{6pt plus 0pt minus 0pt}{6pt plus 0pt minus 0pt}
\titlespacing*{\subsection}{0pt}{18pt plus 64pt minus 0pt}{0pt}
\titlespacing*{\subsubsection}{0pt}{12pt plus 0pt minus 0pt}{0pt}
\titlespacing*{\paragraph}    {0pt}{3.25ex plus 1ex minus .2ex}{1.5ex plus .2ex}
\titlespacing*{\subparagraph} {0pt}{3.25ex plus 1ex minus .2ex}{1.5ex plus .2ex}

% Define the colors of table of contents
% This is helpful to understand http://tex.stackexchange.com/questions/110253/what-the-first-argument-for-lsubsection-actually-is
\definecolor{LochmaraColor}{HTML}{1020A0}

% Hyperlinks
\hypersetup{
    colorlinks = true,
    allcolors = {LochmaraColor},
}


\RequirePackage{tocbibind} %%% comment this to remove page number for following
\addto\captionsenglish{\renewcommand{\contentsname}{Table of contents}}
\addto\captionsenglish{\renewcommand{\listfigurename}{List of figures}}
\addto\captionsenglish{\renewcommand{\listtablename}{List of tables}}
% \addto\captionsenglish{\renewcommand{\chaptername}{Chapter}}




%%reduce spacing for itemize
\usepackage{enumitem}
\setlist{nosep}

%%%%%%%%%%% Quote Styles at the top of chapter
\usepackage{epigraph}
\setlength{\epigraphwidth}{0.8\columnwidth}
\newcommand{\chapterquote}[2]{\epigraphhead[60]{\epigraph{\textit{#1}}{\textbf {\textit{--#2}}}}}
%%%%%%%%%%% Quote for all places except Chapter
\newcommand{\sectionquote}[2]{{\quote{\textit{``#1''}}{\textbf {\textit{--#2}}}}}

% Insert 22pt white space before roc title. \titlespacing at line 65 changes it by -22 later on.
\renewcommand*\contentsname{\hspace{0pt}Contents}


% Define section, subsection and subsubsection font size and color
\usepackage{sectsty}
\definecolor{AllportsColor}{HTML}{A02010}
\allsectionsfont{\color{AllportsColor}}

\usepackage{titlesec}
\titleformat{\section}
{\color{AllportsColor}\LARGE\bfseries}{\thesection.}{1em}{}

\titleformat{\subsection}
{\color{AllportsColor}\Large\bfseries}{\thesubsection.}{1em}{}

\titleformat{\subsubsection}
{\color{AllportsColor}\large\bfseries}{\thesubsubsection.}{1em}{}

\titleformat{\paragraph}
{\color{AllportsColor}\large\bfseries}{\theparagraph}{1em}{}

\titleformat{\subparagraph}
  {\normalfont\normalsize\bfseries}{\thesubparagraph}{1em}{}

\titleformat{\subsubparagraph}
  {\normalfont\normalsize\bfseries}{\thesubsubparagraph}{1em}{}


\title{%
  A LADOK3 Python API
}
\author{%
  Alexander Baltatzis,
  Daniel Bosk,
  Gerald Q.\ \enquote{Chip} Maguire Jr.
}
\affil{%
  KTH EECS\\
  \texttt{\{alba,dbosk,maguire\}@kth.se}
}

\begin{document}
\frontmatter
\maketitle

\vspace*{\fill}
\VerbatimInput{../LICENSE}
\clearpage

\begin{abstract}
  We provide a Python wrapper for the LADOK3 REST API\@.
We provide a useful object-oriented framework and direct API calls that return 
the unprocessed JSON data from LADOK.

\end{abstract}
\clearpage

\tableofcontents
\clearpage

\mainmatter
\chapter{Introduction}

LADOK (abbreviation of \foreignlanguage{swedish}{Lokalt Adb–baserat 
DOKumentationssystem}, in Swedish) is the national documentation system for 
higher education in Sweden.
This is the documented source code of \texttt{ladok3}, a LADOK3 API wrapper for 
Python.

The \texttt{ladok3} library provides a non-GUI application that, similar to 
access via a web browser, only provides the user with access to the LADOK data 
and functionality that they would actually have based on their specific user 
permissions in LADOK.
It can be seem as a very streamlined web browser just for LADOK's web 
interface.
While the library exploits caching to reduce the load on the LADOK server, this 
represents a subset of the information that would otherwise be obtained via
LADOK's web GUI export functions.

You can install the \texttt{ladok3} package by running
\begin{minted}{bash}
pip3 install ladok3
\end{minted}
in the terminal.
You can find a quick reference by running
\begin{minted}[firstnumber=last]{bash}
pydoc ladok3
\end{minted}

We provide the main class \texttt{LadokSession} (\cref{LadokSession}).
The \texttt{LadokSession} class acts like \enquote{factories} and will return 
objects representing various LADOK data.
These data objects' classes inherit the \texttt{LadokData} (\cref{LadokData}) 
and \texttt{LadokRemoteData} (\cref{LadokRemoteData}) classes.
Data of the type \texttt{LadokData} is not expected to change, unlike 
\texttt{LadokRemoteData}, which is.
Objects of type \texttt{LadokRemoteData} know how to update themselves, \ie fetch 
and refresh their data from LADOK.
When relevant they can also write data to LADOK, \ie update entries such as 
results.

One design criteria is to improve performance.
We do this by caching all factory methods of any \texttt{LadokSession}.
The \texttt{LadokSession} itself is also designed to be cacheable: if the session to 
LADOK expires, it will automatically reauthenticate to establish a new session.



\part{The library}

\documentclass[a4paper,oneside]{book}
\newenvironment{abstract}{}{}
\usepackage{abstract}
\usepackage{noweb}
% Needed to relax penalty for breaking code chunks across pages, otherwise 
% there might be a lot of space following a code chunk.
\def\nwendcode{\endtrivlist \endgroup}
\let\nwdocspar=\smallbreak

\usepackage[hyphens]{url}
\usepackage{hyperref}
\usepackage{authblk}

% package with esp-idf specific macros
\usepackage{espidf}

\setcounter{secnumdepth}{2}
\setcounter{tocdepth}{2}

\usepackage{amsmath,amsfonts,amssymb,amsthm}
\usepackage{graphicx}
%%% reduce spaces for Table of contents, figures and tables
%%% it is used "\addtocontents{toc}{\vskip -1.2cm}" etc. in the document
\usepackage[notlot,nottoc,notlof]{}

\usepackage{color}
\usepackage{transparent}
\usepackage{eso-pic}
\usepackage{lipsum}

%%% Needed for displaying Chinese in English documentation
\usepackage{xeCJK}

%% spacing between line
\usepackage{setspace}
\singlespacing


\definecolor{myred}{RGB}{229, 32, 26}
\definecolor{mygrayy}{RGB}{127, 127, 127}
\definecolor{myblack}{RGB}{64, 64, 64}


%%%%%%%%%%% datetime
\usepackage{datetime}

\newdateformat{MonthYearFormat}{%
    \monthname[\THEMONTH], \THEYEAR}


%% RO, LE will not work for 'oneside' layout.
%% Change oneside to twoside in document class
\usepackage{fancyhdr}
\pagestyle{fancy}
\fancyhf{}

% Header and footer
\makeatletter
  \fancypagestyle{normal}{
    \fancyhf{}
    \fancyhead[L]{\nouppercase{\leftmark}}
    \fancyfoot[C]{\py@HeaderFamily\thepage \\ \href{https://www.espressif.com/en/company/documents/documentation_feedback?docId=4287&sections=&version=\idfReleaseName}{Submit Document Feedback}}
    \fancyfoot[L]{Espressif Systems}
    \fancyfoot[R]{\idfReleaseName}
    \renewcommand{\headrulewidth}{0.4pt}
    \renewcommand{\footrulewidth}{0.4pt}
  }
\makeatother

\renewcommand{\headrulewidth}{0.5pt}
\renewcommand{\footrulewidth}{0.5pt}


% Define a spacing for section, subsection and subsubsection
% http://tex.stackexchange.com/questions/108684/spacing-before-and-after-section-titles

\titlespacing*{\section}{0pt}{6pt plus 0pt minus 0pt}{6pt plus 0pt minus 0pt}
\titlespacing*{\subsection}{0pt}{18pt plus 64pt minus 0pt}{0pt}
\titlespacing*{\subsubsection}{0pt}{12pt plus 0pt minus 0pt}{0pt}
\titlespacing*{\paragraph}    {0pt}{3.25ex plus 1ex minus .2ex}{1.5ex plus .2ex}
\titlespacing*{\subparagraph} {0pt}{3.25ex plus 1ex minus .2ex}{1.5ex plus .2ex}

% Define the colors of table of contents
% This is helpful to understand http://tex.stackexchange.com/questions/110253/what-the-first-argument-for-lsubsection-actually-is
\definecolor{LochmaraColor}{HTML}{1020A0}

% Hyperlinks
\hypersetup{
    colorlinks = true,
    allcolors = {LochmaraColor},
}


\RequirePackage{tocbibind} %%% comment this to remove page number for following
\addto\captionsenglish{\renewcommand{\contentsname}{Table of contents}}
\addto\captionsenglish{\renewcommand{\listfigurename}{List of figures}}
\addto\captionsenglish{\renewcommand{\listtablename}{List of tables}}
% \addto\captionsenglish{\renewcommand{\chaptername}{Chapter}}




%%reduce spacing for itemize
\usepackage{enumitem}
\setlist{nosep}

%%%%%%%%%%% Quote Styles at the top of chapter
\usepackage{epigraph}
\setlength{\epigraphwidth}{0.8\columnwidth}
\newcommand{\chapterquote}[2]{\epigraphhead[60]{\epigraph{\textit{#1}}{\textbf {\textit{--#2}}}}}
%%%%%%%%%%% Quote for all places except Chapter
\newcommand{\sectionquote}[2]{{\quote{\textit{``#1''}}{\textbf {\textit{--#2}}}}}

% Insert 22pt white space before roc title. \titlespacing at line 65 changes it by -22 later on.
\renewcommand*\contentsname{\hspace{0pt}Contents}


% Define section, subsection and subsubsection font size and color
\usepackage{sectsty}
\definecolor{AllportsColor}{HTML}{A02010}
\allsectionsfont{\color{AllportsColor}}

\usepackage{titlesec}
\titleformat{\section}
{\color{AllportsColor}\LARGE\bfseries}{\thesection.}{1em}{}

\titleformat{\subsection}
{\color{AllportsColor}\Large\bfseries}{\thesubsection.}{1em}{}

\titleformat{\subsubsection}
{\color{AllportsColor}\large\bfseries}{\thesubsubsection.}{1em}{}

\titleformat{\paragraph}
{\color{AllportsColor}\large\bfseries}{\theparagraph}{1em}{}

\titleformat{\subparagraph}
  {\normalfont\normalsize\bfseries}{\thesubparagraph}{1em}{}

\titleformat{\subsubparagraph}
  {\normalfont\normalsize\bfseries}{\thesubsubparagraph}{1em}{}


\title{%
  A LADOK3 Python API
}
\author{%
  Alexander Baltatzis,
  Daniel Bosk,
  Gerald Q.\ \enquote{Chip} Maguire Jr.
}
\affil{%
  KTH EECS\\
  \texttt{\{alba,dbosk,maguire\}@kth.se}
}

\begin{document}
\frontmatter
\maketitle

\vspace*{\fill}
\VerbatimInput{../LICENSE}
\clearpage

\begin{abstract}
  We provide a Python wrapper for the LADOK3 REST API\@.
We provide a useful object-oriented framework and direct API calls that return 
the unprocessed JSON data from LADOK.

\end{abstract}
\clearpage

\tableofcontents
\clearpage

\mainmatter
\chapter{Introduction}

LADOK (abbreviation of \foreignlanguage{swedish}{Lokalt Adb–baserat 
DOKumentationssystem}, in Swedish) is the national documentation system for 
higher education in Sweden.
This is the documented source code of \texttt{ladok3}, a LADOK3 API wrapper for 
Python.

The \texttt{ladok3} library provides a non-GUI application that, similar to 
access via a web browser, only provides the user with access to the LADOK data 
and functionality that they would actually have based on their specific user 
permissions in LADOK.
It can be seem as a very streamlined web browser just for LADOK's web 
interface.
While the library exploits caching to reduce the load on the LADOK server, this 
represents a subset of the information that would otherwise be obtained via
LADOK's web GUI export functions.

You can install the \texttt{ladok3} package by running
\begin{minted}{bash}
pip3 install ladok3
\end{minted}
in the terminal.
You can find a quick reference by running
\begin{minted}[firstnumber=last]{bash}
pydoc ladok3
\end{minted}

We provide the main class \texttt{LadokSession} (\cref{LadokSession}).
The \texttt{LadokSession} class acts like \enquote{factories} and will return 
objects representing various LADOK data.
These data objects' classes inherit the \texttt{LadokData} (\cref{LadokData}) 
and \texttt{LadokRemoteData} (\cref{LadokRemoteData}) classes.
Data of the type \texttt{LadokData} is not expected to change, unlike 
\texttt{LadokRemoteData}, which is.
Objects of type \texttt{LadokRemoteData} know how to update themselves, \ie fetch 
and refresh their data from LADOK.
When relevant they can also write data to LADOK, \ie update entries such as 
results.

One design criteria is to improve performance.
We do this by caching all factory methods of any \texttt{LadokSession}.
The \texttt{LadokSession} itself is also designed to be cacheable: if the session to 
LADOK expires, it will automatically reauthenticate to establish a new session.



\part{The library}

\documentclass[a4paper,oneside]{book}
\newenvironment{abstract}{}{}
\usepackage{abstract}
\usepackage{noweb}
% Needed to relax penalty for breaking code chunks across pages, otherwise 
% there might be a lot of space following a code chunk.
\def\nwendcode{\endtrivlist \endgroup}
\let\nwdocspar=\smallbreak

\usepackage[hyphens]{url}
\usepackage{hyperref}
\usepackage{authblk}

\input{preamble.tex}

\title{%
  A LADOK3 Python API
}
\author{%
  Alexander Baltatzis,
  Daniel Bosk,
  Gerald Q.\ \enquote{Chip} Maguire Jr.
}
\affil{%
  KTH EECS\\
  \texttt{\{alba,dbosk,maguire\}@kth.se}
}

\begin{document}
\frontmatter
\maketitle

\vspace*{\fill}
\VerbatimInput{../LICENSE}
\clearpage

\begin{abstract}
  \input{abstract.tex}
\end{abstract}
\clearpage

\tableofcontents
\clearpage

\mainmatter
\chapter{Introduction}

LADOK (abbreviation of \foreignlanguage{swedish}{Lokalt Adb–baserat 
DOKumentationssystem}, in Swedish) is the national documentation system for 
higher education in Sweden.
This is the documented source code of \texttt{ladok3}, a LADOK3 API wrapper for 
Python.

The \texttt{ladok3} library provides a non-GUI application that, similar to 
access via a web browser, only provides the user with access to the LADOK data 
and functionality that they would actually have based on their specific user 
permissions in LADOK.
It can be seem as a very streamlined web browser just for LADOK's web 
interface.
While the library exploits caching to reduce the load on the LADOK server, this 
represents a subset of the information that would otherwise be obtained via
LADOK's web GUI export functions.

You can install the \texttt{ladok3} package by running
\begin{minted}{bash}
pip3 install ladok3
\end{minted}
in the terminal.
You can find a quick reference by running
\begin{minted}[firstnumber=last]{bash}
pydoc ladok3
\end{minted}

We provide the main class \texttt{LadokSession} (\cref{LadokSession}).
The \texttt{LadokSession} class acts like \enquote{factories} and will return 
objects representing various LADOK data.
These data objects' classes inherit the \texttt{LadokData} (\cref{LadokData}) 
and \texttt{LadokRemoteData} (\cref{LadokRemoteData}) classes.
Data of the type \texttt{LadokData} is not expected to change, unlike 
\texttt{LadokRemoteData}, which is.
Objects of type \texttt{LadokRemoteData} know how to update themselves, \ie fetch 
and refresh their data from LADOK.
When relevant they can also write data to LADOK, \ie update entries such as 
results.

One design criteria is to improve performance.
We do this by caching all factory methods of any \texttt{LadokSession}.
The \texttt{LadokSession} itself is also designed to be cacheable: if the session to 
LADOK expires, it will automatically reauthenticate to establish a new session.



\part{The library}

\input{../src/ladok3/ladok3.tex}


\part{API calls}

\input{../src/ladok3/api.tex}
\input{../src/ladok3/undoc.tex}



\part{A command-line interface}

\chapter{The base interface}

\input{../src/ladok3/cli.tex}

\chapter{The \texttt{data} command}

\input{../src/ladok3/data.tex}

\chapter{The \texttt{report} command}

\input{../src/ladok3/report.tex}

\chapter{The \texttt{student} command}

\input{../src/ladok3/student.tex}



\part{Other example applications}

\chapter{Transfer results from KTH Canvas to LADOK}

Here we provide an example program~\texttt{canvas2ladok.py} which exports 
results from KTH Canvas to LADOK for the introductory programming course~prgi 
(DD1315).
You can find an up-to-date version of this chapter at
\begin{center}
  \url{https://github.com/dbosk/intropy/tree/master/adm/reporting}.
\end{center}

\input{../examples/canvas2ladok.tex}


\backmatter
\printbibliography


\end{document}



\part{API calls}

\input{../src/ladok3/api.tex}
\input{../src/ladok3/undoc.tex}



\part{A command-line interface}

\chapter{The base interface}

\input{../src/ladok3/cli.tex}

\chapter{The \texttt{data} command}

\input{../src/ladok3/data.tex}

\chapter{The \texttt{report} command}

\input{../src/ladok3/report.tex}

\chapter{The \texttt{student} command}

\input{../src/ladok3/student.tex}



\part{Other example applications}

\chapter{Transfer results from KTH Canvas to LADOK}

Here we provide an example program~\texttt{canvas2ladok.py} which exports 
results from KTH Canvas to LADOK for the introductory programming course~prgi 
(DD1315).
You can find an up-to-date version of this chapter at
\begin{center}
  \url{https://github.com/dbosk/intropy/tree/master/adm/reporting}.
\end{center}

\input{../examples/canvas2ladok.tex}


\backmatter
\printbibliography


\end{document}



\part{API calls}

\input{../src/ladok3/api.tex}
\input{../src/ladok3/undoc.tex}



\part{A command-line interface}

\chapter{The base interface}

\input{../src/ladok3/cli.tex}

\chapter{The \texttt{data} command}

\input{../src/ladok3/data.tex}

\chapter{The \texttt{report} command}

\input{../src/ladok3/report.tex}

\chapter{The \texttt{student} command}

\input{../src/ladok3/student.tex}



\part{Other example applications}

\chapter{Transfer results from KTH Canvas to LADOK}

Here we provide an example program~\texttt{canvas2ladok.py} which exports 
results from KTH Canvas to LADOK for the introductory programming course~prgi 
(DD1315).
You can find an up-to-date version of this chapter at
\begin{center}
  \url{https://github.com/dbosk/intropy/tree/master/adm/reporting}.
\end{center}

\input{../examples/canvas2ladok.tex}


\backmatter
\printbibliography


\end{document}



\part{API calls}

\input{../src/ladok3/api.tex}
\input{../src/ladok3/undoc.tex}



\part{A command-line interface}

\chapter{The base interface}

\input{../src/ladok3/cli.tex}

\chapter{The \texttt{data} command}

\input{../src/ladok3/data.tex}

\chapter{The \texttt{report} command}

\input{../src/ladok3/report.tex}

\chapter{The \texttt{student} command}

\input{../src/ladok3/student.tex}



\part{Other example applications}

\chapter{Transfer results from KTH Canvas to LADOK}

Here we provide an example program~\texttt{canvas2ladok.py} which exports 
results from KTH Canvas to LADOK for the introductory programming course~prgi 
(DD1315).
You can find an up-to-date version of this chapter at
\begin{center}
  \url{https://github.com/dbosk/intropy/tree/master/adm/reporting}.
\end{center}

\input{../examples/canvas2ladok.tex}


\backmatter
\printbibliography


\end{document}
